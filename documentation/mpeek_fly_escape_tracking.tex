\documentclass[12pt]{amsart}
\usepackage{graphicx}
\usepackage[hmargin=1in,vmargin=.75in]{geometry}
\usepackage{amsmath}

\begin{document}

\title{Tracking of Drosophila Leg, Body and Wing Position During Visually-Elicited Escape Behavior}
\author{Martin Peek\\ TTIC 31130 Visual Recognition}
%\date{}

\maketitle{}

\section*{Introduction}

I am interested in the causative relationship between groups of connected neurons and behavior. We can define behavior as the complete set of movements of an intact animal, and roughly classify behaviors as innate and learned. Innate behaviors are those which can be performed without prior experience and are thought to be genetically hard-wired, whereas learned behaviors change over time with experience. I am currently working in the fruit fly, {\it Drosophila melanogaster}, to identify and characterize neurons which form the basis of its innate escape behavior. 

Fly escape can be triggered by an approaching object, like a fly swatter. Escapes have been described in detail in two previous studies by Card and Dickinson\cite{card2008cb} \cite{card2008jeb}. These studies investigated the escape behavior of wild type flies using high speed videography (6,000 fps). Body and wing orientation were determined by manually selecting points and fitting lines on video frames. This approach is time consuming but reasonable for their data set of about 50 flies. The analyses show that flies are sensitive to the direction of an approaching object. In addition, prior to takeoff, they make leg and postural adjustments so that when they jump and flap away, they move in a direction away from the incoming object.

To find the neuronal basis of fly escape behavior, we will need to develop at much larger data set. While Drosophila are only 3mm long, their nervous systems contain on the order of 10,000 neurons. To find neurons relevant to the escape behavior, we use genetic manipulations to disable certain identified neurons in the fly brain and then assay the behavior of these flies with high speed videography. Each video consists of a fly walking onto a prism followed by an escape elicited by a visual stimulus. Analysis of the videos will indicate when shutting down these neurons has affected the behavior and allow us to trace which neurons have behavioral control. We are looking for abnormal escape behavior. To cover a large portion of the fly's brain, we will use more than 1000 genetically modified fly lines with different disabled subsets, producing around 20,000 videos. Currently, a large set of test videos is available for developing a computer vision-based analysis.


\section*{Problem Statement and Project Goals}

I am interested in developing computer vision code that analyzes frames of a high speed video to determine the following features of the fly over time: leg position, wing position, azimuthal body orientation.


\section*{Overview of Planning and Progress}
I would like to use this \LaTeX\ file as a working document to record progress and results. I will use section headings for each week and relevant subsections to describe goals, work, and problems encountered. I will put this and other relevant files in a repository soon.

\section*{Week 3}
\subsection*{To Do}
Try Matlab code from Raquel!
\subsection*{Results and Analysis}
Nothing yet.

\bibliography{refs}
\bibliographystyle{ieeetr}

\end{document}